\chapter{Quantum mechanics}

\section{State of a system}
The instantaneous state of a system is represented by points in a Hilbert space: $|\psi(t)\rangle$.
More precisely,  states are associated to rays - since $a|\psi\rangle$ and $|\psi\rangle$ represent the same state.
Let $|\psi_3\rangle = a |\psi_2\rangle$, $|\psi_1\rangle$ and $|\psi_2\rangle$ represent different states and $|\psi_1\rangle$ and $|\psi_3\rangle$ represent the same one.\\
\noindent
The wavefunction $\psi(Q,t)$ describes the state of a quantum system, where $Q=\vec{q_1},...,\vec{q_N}$ are coordinates for the position of all the particles. Keep in mind that $\psi$ cannot be measured, it is a mere mathematical tool that we use for computing and making predictions.

\section{Observable quantities}
Observable quantities are associated to an Hermitian Operator, for example:

$$\underbrace{x}_{\text{position}}\rightarrow\underbrace{\hat{x}}_{\text{position operator}}$$

$$\underbrace{h}_{\text{energy}}\rightarrow\underbrace{\hat{H}}_{\text{Hamiltonian operator}}$$

The only exception is time, which is not associated to an operator. 

$$O(p,q)\rightarrow\hat{O}(\underbrace{-i\hbar\vec{\nabla}}_{\text{p}},\bar{q})$$

\section{Outcomes of measurement}
The possible outcomes of measurement of an observable $O$ are the eigenvalues of the corresponding operator $\hat{O}$:

$$\hat{O}|o_n\rangle = o_n|o_n\rangle$$

If the system is in the quantum state $|\psi\rangle$, the probability of finding the value $o_n$ when measuring $\hat{O}$ is:

$$P(o_n) = |\langle o_n|\psi\rangle|^2$$

Immediately after the measurement of $\hat{O}$ , yielding $o_n$, the state of the system is described by:
$$psi(Q,t_0)=f_n°(Q)$$
By measuring we observe the \textbf{collapse of the wavefunction} e.g. the state of the cat collapses to either dead or alive when we open the box.
\section{Expectation value}
The average over many measurement of $\hat{O}$ or expectation value is given by:

$$\langle \hat{O}\rangle = \frac{\langle\psi|\hat{O}|\psi\rangle}{\langle \psi|\psi\rangle}=\frac{\int d^{3 N} Q \psi^{*}(Q, t) \hat{O} \psi(Q, t)}{\int d_{Q}^{3 N} \psi^{*}(Q, t) \psi(Q, t)}$$

As a corollary,  the mean-square deviation of the measurement is:

$$\Delta^2\hat{O} = \frac{\langle \psi|\hat{O}^2|\psi\rangle}{\langle\psi|\psi\rangle}-\biggl(\frac{\langle\psi|\hat{O}|\psi\rangle}{\langle\psi|\psi\rangle}\biggr)^2$$

It $|\psi\rangle$ is an eigenstate of $\hat{O}$ $|\psi\rangle = |o_n\rangle$, then $\Delta^2\hat{O} = 0$

\section{Time evolution}
The time evolution of $|\psi(t)\rangle$ is described by the Schr\"odinger equation:

$$i\hbar \frac{d{}}{d{t}}|\psi(t)\rangle = H|\psi(t)\rangle$$

The formal solution of this equation is given by the time evolution operator:

$$|\psi(t)\rangle = \hat{U}(t-t_0(|\psi(t_0)\rangle$$

$$\hat{U}(t-t_0) = e^{-\frac{i}{\hbar}\hat{H}(t-t_0)}$$

\section{Remarks}
$\hat{O}_1,\hat{O}_2,...,\hat{O}_N$ are Hermitian operators.\\
Let ${O_l}_{l=1,...,M<N}$ be the subset of mutually commuting operators. Then, for $[O_l,O_m]=0$ with $O_l,O_m \in M$ there will be a common set of eigenstates.\\
\noindent

$\hat{O}_lf_i=o_i^{(l)}f_i$, where $f_i$ are the eigenstates. By measuring immediately after a previous measurement $M,$ the result will be equal - as the $M$ measurement will provide the maximum information possible on the quantum system. Any other measurement will destroy the information e.g. if we measure the momentum (non-commuting) after the position, we will not know the position anymore.

Example: check whether $L_x$ and $L_y$ do not commute.
$$\bar{L}=\left|\begin{array}{ll}\hat{i} \hat{j} \hat{k} \\ x y z \\ p_{x} p_{y} p_{z}\end{array}\right|=\hat{i} \underbrace{\left(y p_{z}-z p_{y}\right)}_{L_{x}}+\hat{j}(\underbrace{z p_{x}-x p_{z}}_{L_{y}}+\hat{k} \underbrace{\left(x p_{y}-y p_{x}\right)}_{L_{z}}$$

Quantization leads to:
$$\begin{array}{l}
\hat{L}_{x}=-i \hbar\left(y \frac{\partial}{d z}-z \frac{\partial}{\partial y}\right) \quad \hat{L}_{2}=-i \hbar\left(x \frac{\partial}{\partial y}-y \frac{\partial}{\partial x}\right) \\
\hat{L}_{y}=-i \hbar\left(z \frac{\partial}{\partial x}-x \frac{\partial}{\partial z}\right)
\end{array}$$

Find if $\left[\hat{L}_{x}, \hat{L}_{y}\right]=?$:
$$
\underbrace{(-i \hbar)^{2}}_{\text {collect }}\left[\hat{L}_{x} \hat{L}_{y} \varphi-\hat{L}_{y} \hat{L}_{x}\varphi\right\}=
$$
Apply definition:
$$
(-i \hbar)^{2}\left\{\left(y \frac{\partial}{\partial z}-z \frac{\partial}{\partial y}\right)\left[\left(z \frac{\partial}{\partial x}-x \frac{\partial}{\partial z}\right) \varphi(x, y, z)\right]\right\}-(-i h)^{2}\left\{\left(z \frac{\partial}{\partial x}-x \frac{\partial}{\partial z}\right)\left[\left(y \frac{\partial}{\partial z}-z \frac{\partial}{\partial y}\right) \varphi(x, y, z)\right]\right\}
$$
Solve:
$$(-i \hbar)^{2}\left\{\left(y \frac{\partial}{\partial z}-z \frac{\partial}{\partial y}\right)\left[z \frac{\partial}{\partial x} \varphi-x \frac{\partial}{\partial z} \varphi\right]\right\}-(-i h)^{2}\left\{\left(2 \frac{\partial}{\partial x}-x \frac{\partial}{\partial z}\right)\left[y \frac{\partial}{\partial z} \varphi-2 \frac{\partial}{\partial y} \varphi\right]\right\}$$

\begin{multline}
(-i \hbar)^{2}{y \frac{\partial}{\partial x} \varphi+y z \frac{\partial}{\partial z} \frac{\partial}{\partial x} \varphi-y x \frac{\partial^{2}}{\partial z^{2}} \varphi-z^{2} \frac{\partial}{\partial y} \frac{\partial}{\partial x} \varphi+z x \frac{\partial}{\partial y} \frac{\partial}{\partial z} \varphi} \\-(i\hbar)^{2}{z y \frac{\partial}{\partial x} \frac{\partial}{\partial z} \varphi-z^{2} \frac{\partial}{\partial x} \frac{\partial}{\partial y} \varphi-x y \frac{\partial^{2}}{\partial z} \varphi+x \frac{\partial}{\partial y} \varphi+x z \frac{\partial}{\partial z} \frac{\partial}{\partial y} \varphi}
\end{multline}
By deleting terms:
$$=(-i \hbar)(-i \hbar)[y \frac{\partial}{\partial x} \varphi-x\frac{\partial}{\partial y}\varphi] = i\hbar \hat{L}_{z}\varphi$$
As a result, $\left[\hat{L}_{x}, \hat{L}_{y}\right]=i\hbar \hat{L}_{z}$.\\

\noindent
The other relationships are the following:
$$\left[\hat{L}_{z}, \hat{L}_{x}\right]=-i \hbar \hat{L}_{y}$$
$$\left[\hat{L}_{y}, \hat{L}_{z}\right]=-i \hbar \hat{L}_{x}$$
Keep in mind that $[\hat{L^2}, \hat{L}_{x}]=0,[\hat{L^2}, \hat{L}_{y}]=0,[\hat{L^2}, \hat{L}_{z}]=0$.\\
\noindent
Summarizing,  $\hat{L}_{x},\hat{L}_{y},\hat{L}_{z}$ are not mutually commuting, i.e., we can only know one compontent of the orbital momentum and the squared modulus (total length of the orbital momentum) at the same time.

