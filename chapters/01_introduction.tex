\chapter{Revisiting classical mechanics}

\section{Physical theories}

  \subsection{Experiment}
  An experiment performed on a physical system is a way to measure observable quantities at a determined time: $O_1(t), \dots, O_n(t)$.
  By measuring more and more observables at the same time the instantaneous state of the system is more and more characterized.
  A maximum set of observables that leads to the complete characterization of the instantaneous state can be assumed.

    \subsubsection{Example - particle of mass $m$ subject to an harmonic force}
    For a particle of mass $m$ subject to an harmonic force like one of a spring is completely characterized by: 
    
    $$(\vec{r}(t),\vec{v}(t))=\vec{r}(t)\in \mathbb{R}^6$$

  \subsection{Definition}
  A physical theory is a mathematical scheme to predict the state of the system, the outcome of feature observations: $O_1(t'), \dots, O_n(t')$.
  In particular an equation that can be used to compute the state at $t'$ given the state at $t$ is called an equation of motion.

\section{Classical mechanics}

  \subsection{Example - point particle moving in $1D$ and subject to an harmonic force}
  For a point particle moving in $1D$ subject to an harmonic force follows the Newton's law for its equation of motion:

  $$m \frac{d{^2}}{d{t^2}}x(t) = -kx(t)$$

  This is a second-order differential equation such that $\frac{d{^2}}{d{t^2}}f(t) = -\frac{k}{m}f(t)$.
  There are only two solutions:

  \begin{align*}
    f_1(t) = \sin(\omega t) &\Rightarrow \frac{d{^2}}{d{t^2}}f)1(t) = -\omega^2f_1(t)\\
    f_2(t) = \cos(\omega t) &\Rightarrow \frac{d{^2}}{d{t^2}}f)2(t) = -\omega^2f_2(t)\\
  \end{align*}

  Clearly, $f_1(t)$ and $f_2(t)$ are solutions if $\omega^2 = \frac{k}{m}$.
  So the most general solution is:

  $$x(t) = A\cos\sqrt{\frac{k}{m}}t+B\sin\sqrt{\frac{k}{m}}t$$

  To find $A$ and $B$ information about the initial conditions are used.
  Let the initial position $x(0) = x_0$.
  Then 

  $$x(0) = A = x_0$$

  Considering initial velocity: $\frac{d{}}{d{t}}x(t)|_{t=0} = v_0$, then:

  \begin{align*}
    \frac{d{}}{d{t}}x(t) &= -\sqrt{\frac{k}{m}}A\sin\sqrt{\frac{k}{m}}t + \sqrt{\frac{k}{m}}B\cos\sqrt{\frac{k}{m}}t\\
    v_0 &= \sqrt{\frac{k}{m}} B \Rightarrow B = \sqrt{\frac{m}{k}}v_0
  \end{align*}

  So the final solution is:

  $$x(t) = x_0\cos\sqrt{\frac{k}{m}}Bt+\sqrt{\frac{m}{k}}v_0\sin\sqrt{\frac{k}{m}}t$$

  \subsection{Phase-space}
  It is convenient to plot the solutions on the phase space, a plane such that on the $x$-axis there is the position $x$ and on the $y$-axis the momentum $mv$.
  Cauchy's theorem implies that given an n-the order differential equation has exactly $n$ solutions.
  Moreover, given $n$ initial conditions there exists exactly one solution.
  Because of this trajectories in the phase space can never intersect.
  In this way future $x(t)$ and $v(t)$ can be unambiguously predicted.
  So classical mechanics is fully deterministic.

  \subsection{Systems in three dimension and with more than one particle}
  For systems with $D=3$ and for more than one particle the equation of motion is:

  $$\begin{cases}
    m_1 \frac{d{^2}}{d{t^2}}\vec{r}_1(t) = \vec{F}_1(\vec{r}_1(t), \dots,\vec{r}_N(t))\\
    \vdots\\
    m_N \frac{d{^2}}{d{t^2}}\vec{r}_N(t) = \vec{F}_N(\vec{r}_1(t), \dots,\vec{r}_N(t))\\
  \end{cases}$$

  Correspondingly the phase-space is $6N$ dimensional.
  Moreover for the $N$ vector equations there are $N$ scalar ones.

  \subsection{Work and energy}
  Let $\vec{r}$ be a trajectory followed by a particle subject to a force $\vec{F}$.
  The work of the force $\vec{F}$ from point $A$ to point $B$ of the trajectory is defined as:

  $$W_{AB} = \int_{\vec{r}_a}^{\vec{r}_b}d \vec{r}\cdot \vec{F}$$

  The kinetic energy of the particle is instead:

  $$T = \frac{1}{2}mv^2$$

  Work and energy are related:

  \begin{align*}
    \frac{d{}}{d{t}}T = \frac{d{}}{d{t}}\frac{1}{2}mv^2 = \frac{1}{2}m \frac{d{}}{d{t}}v^2 = \frac{1}{2}2m \vec{v}\underbrace{\frac{d{\vec{v}}}{d{t}}}_{\vec{a}} = \vec{v} \vec{F}\\
    \int_{t_0}^{t_f}dt \frac{d{}}{d{t}}T = T_B-T_A = int_{t_i}^{t_f}dt \frac{d{\vec{r}}}{d{t}}\vec{F} = \int_{\vec{r}_A}^{\vec{r}_B} d \vec{r}\cdot \vec{F} = W_{AB}\\
    T_B-T_a = W_{A\rightarrow B}
  \end{align*}

    \subsubsection{Conservative forces}
    For conservative forces the work from point $A$ to point $B$ does not depend on the path followed, in particular, for each paths $1$ and $2$: $W_{AB}^1 = W_{AB}^2$ and:

    $$-W_{AB} = U(\vec{r}_B) - U(\vec{r}_A)$$

    Where $U(\vec{r})$ is the potential energy.
    In one dimension:

    \begin{align*}
      U(r) -U(r_0) = -w_{x_0x} = -int_{x_0}^xdyF(y) \Rightarrow\\
      -\frac{d{}}{d{x}}U(x) = F(x)
    \end{align*}

      \paragraph{Three dimensional case}
  
      $$\begin{cases}
        F_x(x,y,z) = - \frac{\partial {}}{\partial {x}}U(x,y,z)\\
        F_y(x,y,z) = - \frac{\partial {}}{\partial {y}}U(x,y,z)\\
        F_z(x,y,z) = - \frac{\partial {}}{\partial {z}}U(x,y,z)\\
      \end{cases}$$
  
      Or, in short hand notation, let $\vec{r}=(x,y,z)$ and $\vec{\nabla}(\frac{\partial {}}{\partial {x}},\frac{\partial {}}{\partial {y}},\frac{\partial {}}{\partial {z}})$, then:
  
      $$\vec{F}(\vec{r}) = -\vec{\nabla}U(\vec{r})$$
  
      \paragraph{Central forces}
      Central forces are a notable class of conservative forces, for which $\vec{F}(\vec{r}) = \hat{\omega}_{r}f(r)$ and $\vec{r} = \hat{U}_{r}|\vec{r}| = \hat{U}_{r}r$.
      Some examples:

      \begin{multicols}{2}
        \begin{itemize}
          \item Coulomb: $\vec{F}_e = \hat{U}_{\vec{r}}\frac{q_1q_2}{r_{12}^2}$
          \item Gravity: $\vec{F}_G = -\hat{U}_r\frac{M_1M_2}{r_{12}^2}G$
          \item Harmonic $\vec{F} = -\hat{\omega}_r(\vec{r}-\vec{r}_0)$
          \item $\cdots$
        \end{itemize}
      \end{multicols}

  \subsection{Conservation of mechanical energy}
  Reconsidering the relationships between $T$ and $W$:

  $$T_B-T_A = W_{A\rightarrow B} = U_A-U_B$$

  The mechanical energy $H$ can be introduced, such that:

  $$H_A = T_A+U_A = T_B+U_B = H_B$$

  The mechanical energy is conserved in the system if only conservative forces act on it.
  Energy conservation allow to solve Newton's equation generally impossible to handle.
  This is because conservation laws help gaining partial information without having to solve Newton's equation.
  At this level energy conservation comes in as a matter of convenience.

    \subsubsection{Example}
    Consider a cart going down a path with a loop.
    Let $A$ the starting highest point when it starts going and $B$ the lowest point where it stops accelerating.

    $$\begin{cases}H_A = \underbrace{T_A}_{=0}+\underbrace{U_A}_{=mgh}\\
    H_B = \underbrace{T_B}_{-\frac{1}{2}mv^2}+\underbrace{U_B}_{=0}\end{cases}
    \Rightarrow v = \sqrt{2gh}$$

  \subsection{Angular momentum conservation}
  Let the angular momentum:

  $$\vec{L}(t) = \vec{r}(t)\times m \vec{v}(t)$$

  There are two ways to solve a cross product: $\vec{a}\times \vec{b}\perp \vec{a}$, $\vec{a}\times \vec{b}\perp \vec{b}$ and $|\vec{a}\times \vec{b}| = |\vec{a}||\vec{b}|\sin\theta$.
  This implies that $\vec{a}\parallel \vec{b}\Rightarrow \vec{a}\times \vec{b} = 0$.
  Now considering the vectors' coordinates:

  $$\vec{a}\times \vec{b} = \hat{i}(a_yb_z - a_zb_y) + \hat{j}(a_xb_z - a_zb_x) + \hat{k}(a_xb_y-a_yb_x)$$

  Now considering:

  $$\frac{d{}}{d{t}}\vec{L} = \frac{d{}}{d{t}}\vec{r}\times \vec{v}m + \vec{r}xm \vec{a} = \vec{r}\times \vec{F}$$

  If $\vec{r}\parallel \vec{F}\Rightarrow \frac{d{}}{d{t}}\vec{L} = 0$.
  Then for a conservative force there is angular momentum conservation
  So the motion in a centra force conserves energy and angular momentum.

\section{Classical theory of the hydrogen atom}
The classical theory of the hydrogen atom is defined by the classical Bohr model.
The hydrogen atom if formed by a proton in the centre with an electron orbiting around it.
Because $\frac{m_e}{M_p}\approx \frac{1}{2000}$, for the sake of simplicity an infinite proton mass $\frac{m_e}{M_p}=0$ is assumed.
Because $\frac{d{}}{d{t}}\vec{L}=0$ (angular momentum conservation) electron's motion occurs on a plane and it is two dimensional

  \subsection{Mechanical energy in polar coordinates}

  $$H = \frac{1}{2}m v^2-\frac{e^2}{r}$$

  Now:

  $$\vec{v} = \hat{u}_\theta v_\theta +\hat{u}_r\underbrace{v_r}_{\frac{d{r}}{d{t}}}\Rightarrow v_2 = \biggl(\frac{d{r}}{d{t}}\biggr)^2+v_\theta$$

  Now $\vec{L}^2 = (\vec{r}\times m \vec{v})^2 = mv_\theta^2r$, therefore:

  $$H = \frac{1}{2}m\biggl(\frac{d{r}}{d{t}}\biggr) + \underbrace{\frac{L^2}{2mr^2}}_{\text{constant}}-\frac{e^2}{r}$$
  
  This term depends on $r$ and not on $\theta$ and effectively looks like a potential energy.
  So $\frac{L^2}{2mr^2}-\frac{e^2}{r}\equiv V_{eff}(r)$.
  And:

  $$H = \frac{1}{2}m\biggl(\frac{d{r}}{d{t}}\biggr)^2+V_{eff}(r)$$

  So using polar coordinate and angular momentum conservation the mechanical energy has been written in the form of that of an effective one dimensional system with $U(r) \rightarrow V_{eff}(r)$.
  Using this trick it is immediate to infer the qualitative structure of orbits.

  \subsection{Case 1 - $E>0$}
  The case that $E>0$ is the case of an unbound orbit.
  There is an inversion point and:

  $$\begin{cases} H = \frac{1}{2}m\biggl(\frac{d{r}}{d{t}}\biggr)^2+V_{eff}(r)\\
    \parallel\\
    E<V_{eff}(r)
  \end{cases}
  \Rightarrow \biggl(\frac{d{r}}{d{t}}\biggr)^2< 0$$

  And that is impossible.

  \subsection{Case 2 - $E<0$}
  The case that $E<0$ is the case of  the bound orbit.
  
  \subsection{Conclusion}
  Whenever a charge Ganges its velocity it emits $\frac{e}{m}$ radiations.
  The energy loss per unit time is:

  $$P = \frac{2}{3}\frac{m_er_ea^2}{c}$$

  According to Larmer's law.
  The electron would spiral into the nucleus in $10^{-15}s$.
  Because of this classical atoms are unstable.

\section{Hamiltonian formulation of mechanics}
In the Newtonian formulation the fundamental inspect is tat of the force:

$$m \vec{a} = \vec{F}$$

Defining a physical theory in classical mechanics corresponds to specifying what is a force.

  \subsection{Hamilton's theory}
  Hamilton's theory is an equivalent reformulation of mechanics in which the key concept is not the force but the Hamiltonian $H$, which is closely related to energy.
  While from a practical standpoint the two formulation are equivalent with identical equations of motion, modern physics has shown that the notion of energy is more fundamental than the one of force.

  \subsection{Hamilton's equations}
  From this point Energy will be identified in an Hamiltonian:

  $$H(\underbrace{p}_{\text{momentum}}, \underbrace{q}_{\text{position}}) = \frac{p^2}{2m} + U(q)$$

  So the equations of motion becomes:

  $$\begin{cases}
    \dot{q} = + \frac{\partial {H}}{\partial {p}}\Rightarrow \dot{q} = \frac{p}{m}\Rightarrow p = \dot{q}m\\
    \dot{p} = - \frac{\partial {H}}{\partial {q}}\Rightarrow \dot{p} = -\frac{\partial {U(q)}}{\partial {q}}\Rightarrow ma = F
  \end{cases}$$

  The Hamilton's equation describes directly the evolution of a point of phase space and the state of the system.
  Let $\Gamma = (p,q)$ and $H = H(\Gamma)$, then:

  $$\begin{cases}
    \dot{\Gamma}_1 = +\frac{\partial {H}}{\partial {\Gamma_2}}\\
    \dot{\Gamma}_2 = +\frac{\partial {H}}{\partial {\Gamma_1}}
  \end{cases}$$

  For many particles in three dimensions:

  $$\Gamma\overbrace{(\underbrace{\vec{p}_1,\dots,\vec{p}_N}_{\vec{P}\in \mathbb{R}^{3n}};\underbrace{\vec{q}_1,\dots,\vec{q}_N}_{\vec{Q}\in \mathbb{R}^{3N}})}^{\Gamma\in \mathbb{R}^{6N}}$$

  The state of a many body system is described by the evolution of a point in a large dimensional vector space.

  \subsection{Harmonic oscillator}
  Consider $\omega = \sqrt{\frac{k}{m}}$:

  $$\begin{cases}
    p(t) = -\omega q_0\sin\omega t + v_0\cos\omega t\\
    q(t) = q_0\cos\omega t + \frac{v_0}{\omega}\sin\omega t
  \end{cases}$$

  Now from that:

  \begin{align*}
    \frac{p^2}{\omega^2}+q^2 &= q_0^2\sin^2\omega t + \frac{v_0^2}{\omega^2}\cos^2\omega t - 2 \frac{q_0v_0}{\omega}\sin \omega t\cos \omega t + q_0^2\cos^2\omega t + \frac{v_0^2}{\omega_2}\sin^2\omega t + 2\frac{q_0v_0}{\omega}\sin \omega t\cos \omega t =\\
                             &=2\biggl(q_0^2+\frac{v_0^2}{\omega^2}\biggr)
  \end{align*}

  The problem has a general structure of $\frac{q^2}{A}+\frac{p^2}{B}=1$ and the trajectories draw ellipses in the phase space.
