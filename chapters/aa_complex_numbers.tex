\chapter{Complex numbers}
A broad range of problems can be solved within real numbers, however it is easy to find some that are not solvable in $\mathbb{R}$.
For example the equation $x^2+1=0$ has no solution in the real numbers.
Because of this reason, the real number set is extended, trying to develop a coherent framework in which these problems can be solved.
Following this procedure a new variable $i$ is defined, such that:

$$i := \sqrt{-1}\not\in\mathbb{R}$$

This quantity is called the imaginary unit and it is used to define a new kind of numbers or complex numbers, defined in standard form as:

$$z := \underbrace{a}_{\text{Real part, }\Re{z}} + \underbrace{bi}_{\text{Imaginary part, }\Im{z}}$$

Where $a\land b\in\mathbb{R}$.
Thus we can define a new set of numbers $\mathbb{C}$ such that $z\in\mathbb{C}$ and $\mathbb{R}\subset\mathbb{C}$.
In fact any real number is a complex number where $b=0$.

\section{Argand plane}
Complex numbers can be seen as ordered pairs of reals and they can be naturally plotted on the complex or argand plane.
The horizontal direction represents the real axis while the vertical one represents the imaginary one.

\section{Operations}

	\subsection{Addition}
	Let $z, w\in\mathbb{C}$ be two complex numbers such that $z = a+bi$ and $w = c+di$, where $a,b,c,d\in\mathbb{R}$.
	The addition is defined as:

	$$z+w = (a+c) + (b+d)i$$

	\subsection{Subtraction}
	Let $z, w\in\mathbb{C}$ be two complex numbers such that $z = a+bi$ and $w = c+di$, where $a,b,c,d\in\mathbb{R}$.
	The subtraction is defined as:

	$$z-w = (a-c) + (b-d)i$$

	\subsection{Multiplication}
	Let $z, w\in\mathbb{C}$ be two complex numbers such that $z = a+bi$ and $w = c+di$, where $a,b,c,d\in\mathbb{R}$.
	Remembering that $i^2 = -1$, the multiplication of two complex number is:

	\begin{align*}
		z\cdot w &= (a+bi)(c+di) = \\
						 &=ac+adi+bci+bdi^2 =\\
						 &= ac+(ad+bc)i -bd =\\
						 &= (ac-bd)+(ad+bc)i
	\end{align*}

	\subsection{Complex conjugate}
	Let $z\in\mathbb{C}$ be a complex number such that $z=a+bi$, where $a,b\in\mathbb{R}$.
	The complex conjugate is defined as:

	$$z^*=a-bi$$

	So we take the opposite of the imaginary part.

	\subsection{Division}
	Let $z, w\in\mathbb{C}$ be two complex numbers such that $z = a+bi$ and $w = c+di$, where $a,b,c,d\in\mathbb{R}$.
	The complex conjugate can be used to define a division operation that brings the result in standard form.
	The operation is similar to the rationalization of a fraction: the nominator and the denominator are multiplied by the complex conjugate of the denominator.
	This is because the product of a complex number and its conjugate is always real.
	So the division is defined as:

	\begin{align*}
		\frac{z}{w} &= \frac{a+bi}{c+di} = \\
								&=\frac{a+bi}{c+di}\frac{c-di}{c-di}=\\
								&=\frac{ac - adi + bci +bd}{c^2+d^2}=\\
								&= \frac{(ac + bd) + (bc - ad)i}{c^2+d^2} = \\
								&=\frac{ac +bd}{c^2+d^2} + \frac{bc-ad}{c^2+d^2}i
	\end{align*}

\section{Polar form}

	\subsection{Complex numbers as vectors}
	Complex numbers can be plotted as points in the Argand plane, using as coordinates the real and the imaginary parts.
	In this way a complex number can be seen as a vector of modulus:

	$$\rho = |z| = \sqrt{a^2+b^2}$$

	Due to Pitagora's theorem.
	Complex number are added and subctrated as such.

	\subsection{Definition}
	The polar form is useful to have a simple interpretation of multiplication and division and it is defined as:

	$$z :=\rho(\cos\theta + i\sin\theta)$$

	The variables used for this representation are the modulus $\rho$ and the argument $\theta$, the angle between the positive direction of the real axis and the vector itself.
	The modulus of a complex number is always positive.
	Complex numbers in polar form are periodic with the argument $\theta$ with periodicity $2k\pi$, $\forall k\in\mathbb{Z}$.

	\subsection{Conversion between polar form and standard form}
	Any complex number written in standard form can be written in polar form, where:

	$$\begin{cases}
		\theta = \arctan \frac{b}{a}\\
		\rho = \sqrt{a^2 + b^2}
	\end{cases}$$

	Likewise, you can pass from polar form to standard form using:

	$$\begin{cases}
		a = \rho\cos\theta\\
		b = \rho\sin\theta
	\end{cases}$$
   
	\subsection{Operations}

		\subsubsection{Multiplication}
		Let $z, w\in\mathbb{C}$ be two complex numbers such that $z = \rho_z(\cos\theta_z + i\sin\theta_z)$ and $w = \rho_w(\cos\theta_w + i\sin\theta_w)$.
		The multiplication between $w$ and $z$ is:

		\begin{align*}
			zw &= \rho_z\rho_w(\cos\theta_z +i \sin\theta_z)(\cos\theta_w +i \sin\theta_w) =\\
				 &= \rho_z\rho_w[\cos\theta_z\cos\theta_w - \sin\theta_z\sin\theta_w + i(\sin\theta_z\cos\theta_w + \cos\theta_z\sin\theta_w)]
		\end{align*}

		Using now the addition formulas for cosine and sine:

		$$zw = \rho_z\rho_q[\cos(\theta_z+\theta_w) + i\sin(\theta_z+\theta_w)]$$

		\subsubsection{Division}
		Let $z, w\in\mathbb{C}$ be two complex numbers such that $z = \rho_z(\cos\theta_z + i\sin\theta_z)$ and $w = \rho_w(\cos\theta_w + i\sin\theta_w)$.
		In a similar way as the multiplication, the division will be:

		$$\frac{z}{w} = \frac{r_z}{r_w}[\cos(\theta_z-\theta_w) + i\sin(\theta_z-\theta_q)]$$

		\subsubsection{Power}
		According to the de Moivre theorem, for every $n\in\mathbb{N}$ positive integer and $z\in\mathbb{C}$, where $z = \rho(\cos\theta + i\sin\theta)$:

		$$z^n = \rho^n(\cos n\theta + i\sin n\theta)$$

		\subsubsection{N-th root}
		For every $n\in\mathbb{N}$ positive integer and $z\in\mathbb{C}, z = \rho(\cos\theta + i\sin\theta)$:

		$$\sqrt[n]{z} = \rho^{\frac{1}{n}}\biggl[\cos\biggl(\frac{\theta+2k\pi}{n}\biggr) + i\sin\biggl(\frac{\theta+2k\pi}{n}\biggr)\biggr]$$

		Where $k$ is an integer.
		Note that $k$ and $k=k+n$ produce identical solutions, so $k$ can be limited to the set $\{0,1,\dots, n-1\}$.
		In conclusion there are $n$ distinct roots, each with modulus $r^{\frac{1}{n}}$, that when represented together on the Argand plane lie on the circle of radius $\rho$. The roots are equally spaced and they create a regular polygon when joined.

\section{Complex valued functions}
Real functions can be extended to complex valued functions.
Taken $f$ from an interval $A\subset\mathbb{R}$ to $\mathbb{C}$, the function can then be written as:

$$f(x) = u(x) + v(x)i$$

Where $u$ and $v$ are real valued functions.
The limit of a complex valued function exists if the limits of the real and the complex component exist.
    
    \subsection{Complex conjugate}
    The complex conjugate of a composite function propagates through each of the components, meaning you have to change the sign before each imaginary unit $i$. For example:

    $$(\sin(ix) + ie^{3x + 2i})^* = (\sin(ix))^* + (ie^{3x + 2i})^* = \sin(-ix) - ie^{3x - 2i}$$

	\subsection{Derivative}
	The derivative of a complex valued function is obtained differentiating its real and imaginary parts:

	$$f'(x) = u'(x) + v'(x)$$

	The properties of the derivatives can be extended to this case: if $f$ and $g$ are two complex valued functions differentiable at some point $x_0$ in the domain of both functions, $f\pm g$, $fg$ and $\frac{f}{g}$ ($g(x_0)\neq 0$) are differentiable and the values of these functions are, just like in the real case:

	$$(f\pm g)'(x_0) = f'(x_0) + g'(x_0)$$

	$$(fg)'(x_0) = f'(x_0)g(x_0) + f(x_0)g'(x_0)$$

	$$\frac{f}{g}'(x_0) = \frac{f'(x_0)g(x_0)-f(x_0)g'(x_0)}{g^2(x_0)}$$

\section{Complex exponential}
Due to its properties and applications it is desirable to extend the exponential function to the complex field.
A complex exponential function is in the form $e^{a+bi}$.
From the case $a = 0$:

$$e^{ti} = \cos t + i\sin t$$

If $a, b\neq 0$;

\begin{align*}
	e^{a+bi} &= e^ae^{bi} =\\
					 &=e^a(\cos b + i\sin b)
\end{align*}

	\subsection{Properties}
	Not only the product of two complex exponentials meets the classical properties of the real exponentials, the derivatives maintains them too.
	Let $t\in\mathbb{R}$ and $y(t) = e^{(a+bi)t} = e^{at}(\cos t b + i\sin b)$, its derivative with respecto to $t$ is:

	\begin{align*}
		\frac{dy(t)}{dt} &= \frac{de^{(a+bi)t}}{dt}=\\
										 &= (a+bi)e^{(a+bi)t}
	\end{align*}

	It can be demonstrated that given $z\in\mathbb{C}$, $\frac{de^z}{dz} = e^z$.

	\subsection{Roots of a complex number}
	The complex exponential allows to write the $n$ roots of a complex number $z = r(\cos\theta + i\sin\theta)$ as:

	$$w_k = r^\frac{1}{n}e^{i\frac{\theta+2kn}{n}}$$

	Where $k\in\{0, 1, \dots, n-1\}$.
