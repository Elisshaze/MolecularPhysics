\chapter{Differential operators}
Given the concept of partial derivatives new differential operators can be defined to be applied to scalar functions or vectorial fields.

\section{Gradient, divergence and curl}

	\subsection{Gradient}
	Let $f:\mathbb{R}^3\supseteq A\rightarrow\mathbb{R}$, with $A$ an open set of $\mathbb{R}^2$, a function of three variables $f(x,y,z)$.
	Be the function derivable.
	The gradient of $f$ in Cartesian coordinates is defined by:

	$$grad f = \nabla f := \frac{\partial f(x,y,z)}{\partial x}\hat{i} + \frac{\partial f(x,y,z)}{\partial y}\hat{j} + \frac{partial f(x,y,z)}{\partial z}\hat{k}$$

	The gradient is then the vector that takes as components along the axes directions the first order partial derivatives.
	The gradient is the vector of major increment of the function with respect to the variations in the variables.

	\subsection{Divergence}
	Suppose now to have a derivable vectorial field $\vec{V}:\mathbb{R}^3\supseteq A \rightarrow\mathbb{R}^3$, with $A$ an open set of $\mathbb{R}^3$.
	This vectorial field is defined by means of its components along the axis directions: $\vec{V} = v_1\hat{i} + v_2\hat{j} + v_3\hat{k}$.
	It associates a vector to each point of $A$.
	The gradient, if it exists and if it is derivable is a vectorial field.
	The divergence of $\vec{V}$ in Cartesian coordinates is:

	$$div\vec{V} = \nabla\cdot\vec{V} := \frac{\partial v_1(x,y,z)}{\partial x} + \frac{\partial v_2(x,y,z)}{\partial y} + \frac{\partial v_3(x,y,z)}{\partial z}$$

	This gives informations on where a vectorial fields has source or sink, or, when the vectorial field represent a fluid flux, if the fluid is incompressible or solenoidal.

	\subsection{Curl}
	Suppose now to have a derivable vectorial field $\vec{V}:\mathbb{R}^3\supseteq A \rightarrow\mathbb{R}^3$, with $A$ an open set of $\mathbb{R}^3$.
	The application of the curl to $\vec{V}$ is:

	$$rot\vec{V} = \nabla\times\vec{V} := \biggl(\frac{\partial v_3}{\partial y} - \frac{\partial v_2}{\partial z}\biggr)\hat{i} + \biggl(\frac{\partial v_1}{\partial z} - \frac{\partial v_3}{\partial x}\biggr)\hat{j} + \biggl(\frac{\partial v_2}{\partial x} - \frac{\partial v_1}{\partial y}\biggr)\hat{k}$$

	The application of the curl to a vectorial field gives information on if the field rotates around a point and the verse of that rotation.
	The output of the curl is the modulus of the rotation, and the direction is linked by means of the right hand rule to the verse of rotation.

	\subsection{Properties}

	\begin{multicols}{3}
		\begin{itemize}
			\item $\nabla\times\nabla f = \vec{0}$
			\item $\nabla\cdot\nabla\times\vec{V} = 0$
			\item $\nabla\cdot f = \nabla^2 f$
		\end{itemize}
	\end{multicols}

	\subsection{Laplacian}
	The Laplacian is the last operator and it is defines as:

	$$\nabla^2 f = \frac{\partial^2}{\partial x^2}f + \frac{\partial^2}{\partial y^2}f + \frac{\partial^2}{\partial z^2}f$$

\section{Hessian matrix: maxima and minima}
A matrix of second partial derivatives can be build to study functions from $\mathbb{R}^n$ to $\mathbb{R}$.
The presence of possible extrema at one point is linked to the one of a null gradient in that point.
Let $f:\mathbb{R}^2\supseteq A\rightarrow{R}$, with $A$ an open set of $\mathbb{R}^2$ a function $f(x,y)$.
If $(x_0, y_0)$ is a local extremum if exists $\nabla f = \vec{0}$.
To find local extrema are considered points in which $\nabla f = \vec{0}$.
Let $f:\mathbb{R}^2\supseteq A\rightarrow{R}$, with $A$ an open set of $\mathbb{R}^2$ and $f\in C^2$, where $c^2$ meaning that all partial second derivatives exist and are continuous.
Then the hessian matrix is defined as:

$$H_f(x_0, y_0) = \begin{pmatrix} \partial_{xx} f(x_0, y_0) & \partial_{yx}f(x_0, y_0) \\ \partial_{xy} f(x_0, y_0) & \partial_{yy}f(x_0, y_0)\end{pmatrix}$$

This matrix is useful to determine the nature of the extrema.
Let $f\in C^2$ and $(x_0, y_0)$ a critical point of $f$, then:

\begin{multicols}{2}
	\begin{itemize}
		\item If the determinant of $H_f(x_0, y_0) >0$ and $\partial_{xx}f(x_0, y_0) > 0$ then $(x_0, y_0)$ is a minimum.
		\item If the determinant of $H_f(x_0, y_0) >0$ and $\partial_{xx}f(x_0, y_0) < 0$ then $(x_0, y_0)$ is a maximum.
		\item If the determinant of $H_f(x_0, y_0)<0$ then $(x_0, y_0)$ is a saddle point.
		\item If the determinant of $H_f(x_0,y_0) = 0$ further analysis is necessary.
	\end{itemize}
\end{multicols}

\section{Jacobian matrix}
With a vectorial function the concept of gradient can be extended and applied to each component of the function.
Let $f:\mathbb{R}^2\supseteq A\rightarrow\mathbb{R}^2$ be a function of two variable $f = (f_1(x, y), f_2(x, y))$ for whigh all the derivatives exist ant are continuous.
The jacobian matrix is defined as:

$$J_f = \begin{pmatrix} \partial_x f_1 & \partial_y f_1\\\partial_x f_2 & \partial_y f_2\end{pmatrix} = \begin{pmatrix} \nabla f_1\\\nabla f_2\end{pmatrix}$$

Due to the fact that we have a function that takes as input two variables and gives as output two variables, this can be thought as a change of coordinates:

$$f = (f_1(x, y), f_2(x, y)) = (u, w)\qquad (x, y)\rightarrow (u, w)$$

The Jacobian matrix allows to determine the domain of the transformation.
The change of variables is $1\leftrightarrow 1$ (a bijective function) only if the determinant of the Jacobian matrix is not null.
Also the Jacobian determinant makes possible to consistently define the change of volume in changing the coordinates.
Given a transformation from $(x, y)$ to $(u, v)$ the change in the expression of the area with respect to the coordinates is:

$$dA = dxdy = \biggl| det\frac{\partial(x, y)}{\partial(u, v)}\biggr|dudv$$

Where $dx, dy, du, dv$ are small coordinates intervals.

\section{Chain rule}
Let $u(x, y)$ be a differentiable function of two variables that are differentiable function of two variables each $x(s, t)$ and $y(s, t)$, then the composite function is differentiable and the partial derivatives are:

$$\frac{\partial u}{\partial s} = \frac{\partial u}{\partial x}\frac{\partial x}{\partial s} + \frac{\partial u}{\partial y}\frac{\partial y}{\partial s}\qquad\frac{\partial u}{\partial t} = \frac{\partial u}{\partial x}\frac{\partial x}{\partial t} + \frac{\partial u}{\partial y}\frac{\partial y}{\partial t}$$

	\subsection{Gradient in polar coordinates}
	Suppose to have $g(x, y)$ a function of two variables in Cartesian coordinates.
	If $(\rho, \theta)$ are the usual polar coordinates related to $(x, y)$ by $x = \rho\cos\theta$ and $y =\rho\sin\theta$, then by substituting these for $x$ and $y$ $g$ becomes a function of $\rho$ and $\theta$:

	$$g(x, y) = f(\rho(x, y), \theta(x, y))$$

	With $\rho(x,y) =\sqrt{x^2+y^2}$ and $\theta(x, y) = \arctan\frac{y}{x}$.
	The objective is to compute the gradient $\nabla g(x,y)$ and express it in terms of $\rho$ and $\theta$.
	The chain rule can be used to compute the partial derivatives of $g$ with respect to $x$ and $y$:

	$$\frac{\partial g}{\partial x} = \frac{\partial f}{\partial \rho}\frac{\partial \rho}{\partial x} + \frac{\partial f}{\partial \theta}\frac{\partial \theta}{\partial x}\qquad\frac{\partial g}{\partial y} = \frac{\partial f}{\partial \rho}\frac{\partial \rho}{\partial y} + \frac{\partial f}{\partial \theta}\frac{\partial \theta}{\partial y}$$

	To fill these relation there is a need to compute:

	\begin{align*}
		\frac{\partial\rho}{\partial x} &= \frac{\partial\sqrt{x^2+y^2}}{\partial x} = \frac{1}{2}\frac{2x}{\sqrt{x^2}{y^2}} = \frac{x}{\sqrt{x^2+y^2}} = \frac{\rho\cos\theta}{\rho} = \cos\theta\\
		\frac{\partial\rho}{\partial y} &= \frac{\partial\sqrt{x^2+y^2}}{\partial y} = \frac{1}{2}\frac{2y}{\sqrt{x^2}{y^2}} = \frac{y}{\sqrt{x^2+y^2}} = \frac{\rho\sin\theta}{\rho} = \sin\theta\\
		\frac{\partial\theta}{\partial x} &= \frac{\partial\arctan\frac{y}{x}}{\partial x} = -\frac{y}{x^2+y^2} = -\frac{\rho\sin\theta}{\rho^2} = -\frac{\sin\theta}{\rho}\\
		\frac{\partial\theta}{\partial y} &= \frac{\partial\arctan\frac{y}{x}}{\partial y} = \frac{x}{x^2+y^2} = \frac{\rho\cos\theta}{\rho^2} = \frac{\cos\theta}{\rho}\\
	\end{align*}

	So:

	$$\frac{\partial g}{\partial x} = \frac{\partial f}{\partial\rho}\cos\theta + \frac{\partial f}{\partial\theta}\frac{-\sin\theta}{\rho}\qquad \frac{\partial g}{\partial y} = \frac{\partial f}{\partial \rho}\sin\theta + \frac{\partial f}{\partial \theta}\frac{\cos\theta}{\rho}$$

	The gradient of $g$ using Cartesian versors will be:

	\begin{align*}
		\nabla g = g_x\hat{e}_x + g_y\hat{e}_y &= \biggl(\frac{\partial f}{\partial\rho}\cos\theta + \frac{\partial f}{\partial \theta}\frac{-\sin\theta}{\rho}\biggr)\hat{e}_x +\biggl(\frac{\partial f}{\partial\rho}\sin\theta + \frac{\partial f}{\partial \theta}\frac{\cos\theta}{\rho}\biggr)\hat{e}_y=\\
																					 &=\biggl(\frac{\partial f}{\partial\rho}\cos\theta\biggr)\hat{e}_x + \biggl(\frac{\partial f}{\partial \theta}\frac{-\sin\theta}{\rho}\biggr)\hat{e}_x +\biggl(\frac{\partial f}{\partial\rho}\sin\theta\biggr)\hat{e}_y + \biggl(\frac{\partial f}{\partial \theta}\frac{\cos\theta}{\rho}\biggr)\hat{e}_y=\\
																					 &=\frac{\partial f}{\partial\rho}(\cos\theta\hat{e}_x + \sin\theta\hat{e}_y) + \frac{1}{\rho}\frac{\partial f}{\partial\theta}(-\sin\theta\hat{e}_x + \cos\theta\hat{e}_y)
	\end{align*}

	The unit versors $\hat{e}_\rho$ and $\hat{e}_\theta$ are introduced.
	They have unitary modulus and direction that change from point to point.
	In particolar for polar coordinates they have the components $\hat{e}^\rho = (\cos\theta, \sin\theta)$ and $\hat{e}_\theta = (-\sin\theta, \cos\theta)$.
	So finally the explicit gradient in polar coordinates is:

	$$\nabla g = f_\rho\hat{e}_\rho + \frac{f_\theta}{\rho}\hat{e}_\theta$$
