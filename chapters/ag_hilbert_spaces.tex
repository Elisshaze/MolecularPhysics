\chapter{Hilbert spaces}

\section{From vector to hilbert spaces}

	\subsection{Definition}

		\subsubsection{Classical mechanics}
		In classical mechanics the instantaneous state of a single particle is specified by the vector position $\vec{r}(t)$ and its momentum $\vec{p}(t)$ in the real vector spaces.
		In these vector spaces linearity holds: any linear combination of elements arbitrarily chosen inside the vector space $\mathcal{V}$ is still an element of the same vector space:

		$$\forall \vec{v}, \vec{w}\in\mathcal{V}\land\forall a, b\in\mathbb{R}: a\vec{v}+b\vec{w}\in\mathcal{V}$$

		Imposing that $a,b\in\mathbb{R}$ the discussion is restricted to real vector spaces.
		These have an operation called inner or scalar product that takes as input two vectors and gives as output a scalar.
		This is a real number for real vector spaces:

		$$(\vec{v}, \vec{w})\equiv \vec{v}\cdot\vec{w} = \vec{w}\cdot\vec{v}\in\mathbb{R}$$

		This operation is bi-linear:

		$$(a_1\vec{v}_1 + a_2\vec{v}_2)\cdot(b_1\vec{w}_1 + b_2\vec{w}_2) = \sum\limits_{i, k=1}^2 a_ib_k\vec{v}_i\vec{w}_k$$

		\subsubsection{Quantum mechanics}
		In quantum mechanics the state of a particle is instantaneously described by a quantum state $|\psi(t)\rangle$ that belongs to a hilbert space or $\mathcal{H}$, which provides a suitable generalization for the notion of vector speces.
		In a hilbert space any linear combination of quantum states is still inside the hilbert space:

		$$\forall |\psi\rangle,|\phi\rangle\in\mathcal{H}\land \alpha,\beta\in\mathbb{C}: \alpha|\psi\rangle + \beta|\phi\rangle\in\mathcal{H}$$

		A scalar product can be defined:

		$$\langle\psi|\phi\rangle = (\langle\phi|\psi\rangle)^*\in\mathbb{C}$$

		In hilbert spaces reversing the order of elements in the inner spaces leads to the complex conjugate result.
		The inner product is bi-linear:

		$$\begin{cases}|\xi\rangle \equiv \alpha_1|\psi_1\rangle + \alpha_2|\psi_2\rangle\\|\omega\rangle \equiv \beta_1|\phi_1\rangle + \beta_2|\phi_2\end{cases}\Rightarrow \langle\xi|\omega\rangle = \sum\limits_{i=1}^2\alpha_i^*\beta_i\langle \psi_i|\phi_i\rangle$$

	\subsection{Complete orthonormal bases}

		\subsubsection{Standard vector spaces}
		In standard vector spaces orthonormal basis vectors are a set form of elements which are mutually orthogonal and have unitary norm:

		$$\vec{e}_i\cdot\vec{e}_j = \delta_{ij}$$

		Where $\delta_{ij}$ is the kronecker-delta and is defined as:

		$$\delta_{ij} = \begin{cases}1\qquad i = j\\ 0\qquad i\neq j\end{cases}$$

		It also has the property that:

		$$\sum\limits_j\delta_{ij}A_j = A_i$$

		A set of $N$ versors is said to form a complete orthonormal basis of the vector space $\mathcal{V}$ if for any vector $\vec{v}\in\mathcal{V}$ there exists a unique set of $N$ real coefficient $\lambda_1, \dots, \lambda_N$ which enable to express $\vec{v}$ as a linear combination of the basis:

		$$\vec{v} = \sum\limits_{k=1}^N\lambda_k\vec{e}_k$$

		$N$ is the dimension of the vector space.
		The set of coefficients is called the coordinates of the vector $\vec{v}$ in the given complete orthonormal basis $\{e_k\}_{k = 1, \dots, N}$.

		\subsubsection{Hilbert space}
		A set of quantum states is defined to be a complete orthonormal basis of the Hilbert space $\mathcal{H}$ if:
		\begin{multicols}{2}
			\begin{itemize}
				\item They are mutually orthogonal.
				\item They have unit norm.
				\item Any state can be written as linear combination of them.
			\end{itemize}
		\end{multicols}

		The elements of a complete orthonormal basis of a Hilbert space may form an infinite and dense set.
		Let's consider the set of position quantum states $|x\rangle$.
		Clearly two position can differ by an infinitesimal amount, therefore a continuous index $x$ is needed to label them.
		Two position states are said to obey the orthonormally condition if if holds:

		$$\langle \vec{x}|\vec{y}\rangle = \delta(\vec{x}-\vec{y}_)$$

		$\delta(\vec{x}-\vec{y})$ denotes the dirac-delta:

		$$\int d^3\vec{y} A(y)\delta(\vec{x}-\vec{y}) = A(\vec{x})$$

		The fact that position states form a basis of $\mathcal{H}$ expresses the fact that any quantum state in the Hilbert space can be obtained from a linear combination of the position states:

		$$|\psi\rangle = \int d^3 \vec{x}\phi(\vec{x})|\vec{x}\rangle\qquad \phi(\vec{x})\in\mathbb{C}$$

		The complex function $\phi(x)$ is called the wave function and can be regarded as a dense and infinite set of complex coefficients.
		Therefore Hilbert spaces are infinite dimension vector spaces.

	\subsection{Operators}
	Operators are defined by their action on the elements of the vector space:

	$$\vec{w} = \hat{O}\vec{v}$$

	In particular $\hat{O}$ is linear if:

	$$\hat{O}(\alpha_1\vec{v}_1 + \alpha_2\vec{v}_2) = \alpha_1\hat{O}\vec{v}_1 + \alpha_2\hat{O}\vec{v}_2$$

	Once a basis of a $N$-dimensional real vector space is defined, then each linear operator can be assigned a $N\times N$ matrix, through the representation of the operator in the specific basis:

	$$\vec{w} = \hat{O}\vec{v}\rightarrow \vec{e}_j\cdot\vec{w} = \sum\limits_{i=1}^N(\vec{e}_i\cdot\vec{v})\vec{e}_j\cdot\hat{O}\vec{e}_i = \sum\limits_{i=1}^NO_{ji}v_j$$

	Where $v_i = \vec{e}_i\cdot \vec{v}$ and $O_{ij} = \vec{e}_i\cdot\hat{O}\vec{e}_j$.
	In a complete analogy, a linear operator $\hat{O}$ defined in an hilbert space $\mathcal{H}$ linearly maps a quantum state into another:

	$$|w\rangle = \hat{O}|v\rangle\qquad \hat{O}(\alpha_1|v_1\rangle + \alpha_2|v_2\rangle) = \alpha_1\hat{O}|v_1\rangle +  \alpha_2\hat{O}|v_2\rangle$$

	Like real vector space, operators in Hilbert spaces can be represented in a given orthonormal basis like the position state basis through a projection procedure:

	$$|\omega\rangle = \hat{O}|\psi\rangle\Rightarrow \omega(\vec{x} = \langle\vec{x}|w\rangle) = \int d^3\vec{y}\langle\vec{y}|\psi\rangle\langle\vec{x}|\hat{O}|\vec{y}\rangle = \int d^3\vec{y} O(\vec{x}, \vec{y})\psi(\vec{y})$$

	Where $\omega(\vec{x})$ and $\psi(\vec{x})$ denote the wave functions associated to the states $|\omega\rangle$ and $|\psi\rangle$ respectively.
	In most cases $O(\vec{x}, \vec{y})$ is a nearly local operator.

		\subsubsection{Multiplicative operator}
		A multiplicative operator is the potential energy operator: $U(\vec{x}, \vec{y}) = \delta(\vec{x}-\vec{y})u(y)$:

		$$\int d^3\vec{y}U(\vec{x}, \vec{y})\psi(\vec{y}) = u(\vec{x})\psi(\vec{x})$$.

		\subsubsection{Derivative operator}
		A derivative operator is the kinetic energy operator: $T(\vec{x}, \vec{y}) = -\frac{\hbar^2}{2m}\delta(\vec{x}-\vec{y})\nabla^2_{\vec{x}}$:

		$$\int d^3\vec{y}T(\vec{x}, \vec{y})\psi(\vec{y}) = -\frac{\hbar^2}{2m}\nabla^2\psi(\vec{x})$$

		\subsubsection{Exponential of an operator}
		Let $\hat{A}$ an operator, then:

		$$e^{\hat{A}} \equiv\mathbb{1} + \hat{A} + \frac{1}{2}\hat{A}^2+\cdots = \sum\limits_{n=0}^\infty\frac{1}{n!}\hat{A}^n$$

		Not that:

		$$e^{\hat{A} + \hat{B}} \neq e^{\hat{A}}e^{\hat{B}}$$

\section{Spectral theorem}
The spectral theorem is a fundamental result in the theory of linear operators in vector and hilbert spaces and specifies the general conditions under which operators can be diagonalized to yield a complete orthonormal basis.
The spectral theorem of standard linear algebra follows as a special case of this fundamental result.

	\subsection{Adjoint}
	The adjoint or hermitian conjugate operator $O^\dagger$ of a generic linear operator $\hat{O}$ is defined as:

	$$\langle \vec{v}\hat{O}^\dagger, \vec{w}\rangle = \langle\vec{v}, \hat{O}\vec{w}\rangle$$

	An operator is called hermitian if it is self-adjoint, or if it coincides with its hermitian conjugate: $\hat{O} = \hat{O}^\dagger$.
	Furthermore an operator $\hat{U}$ is called unitary if $\hat{U}^\dagger\hat{U} = \mathbb{1}$.
	Observables in quantum mechanics are associated to hermitian operators and the outcome of a measurement it's one of its real eigenvalues.
	The expectation value of an observable $O$ in any state $\phi$ is:

	$$\frac{(\phi,O,\phi)}{(\phi,\phi)}$$

	\subsection{Statement of the spectral theorem}
	Let $\hat{O}$ be an hermitian operator defined on a hilbert space $\mathcal{H}$.
	Then there exist a complete orthonormal basis of $\mathcal{H}$ defined by the eigenstates of $\hat{O}$.
	Furthermore each eigenvalue is real.

	\subsection{Corollaries}

		\subsubsection{First corollary}
		Hermitian matrices are such that:

		$$(O^T)^* = O$$

		\subsubsection{Second corollary}
		Hermitian matrices in real vector spaces are symmetric.

		\subsubsection{Third corollary}
		Given a complete orthonormal basis of a hilbert space $\{|e_n\rangle\}$ possibly dense and a hermitian operator $\hat{O}$, it is possible to identify a unitary transformation which connects the $\{|e_n\rangle\}$ with the basis of eigenstates of $\hat{O}$, $\{|o_n\rangle\}$.

\section{Fourier transform}
A special case of basis change is provided by the fourier transformation.
Let $\phi(\vec{x})$ be the wave function in coordinate representation.
The unitary transformation to the momentum basis is called direct fourier transform and is defined as:

$$\hat{F}[\phi(\vec{x})] = \tilde{\phi}(\vec{p}) = \int\limits_{-\infty}^\infty d^3\vec{x}e^{\frac{i}{\hbar}\vec{p}\cdot\vec{x}}\phi(\vec{x})$$

The inverse transformation from the momentum basis to the position one is called the inverse fourier transform:

$$\hat{F}^{-1}[\tilde{\phi}(\vec{p})] = \phi(\vec{x}) = \int\limits_{-\infty}^\infty \frac{d^3\vec{p}}{(2\pi)^3}e^{-\frac{i}{\hbar}\vec{p}\cdot\vec{x}}\tilde{\phi}(\vec{p})$$

The term $(2\pi)^3$ is introduced conventionally to guarantee the preservation of the normalization condition.
An important properties of the fourier transform is that:

\begin{align*}
	F^{-1}[\vec{p}\tilde{\phi}(\vec{p})] &= \int\limits_{-\infty}^\infty \frac{d^3\vec{p}}{(2\pi)^3}e^{-\frac{i}{\hbar}\vec{p}\cdot\vec{x}}\vec{p}\tilde{\phi}(\vec{p})=\\
					     & = \biggl(-i\hbar\frac{\partial}{\partial \vec{x}}\biggr)\int\limits_{-\infty}^\infty \frac{d^3\vec{p}}{(2\pi)^3}e^{-\frac{i}{\hbar}\vec{p}\cdot\vec{x}}\tilde{\phi}(\vec{p})\\
					     & = \biggl(-i\hbar\frac{\partial}{\partial \vec{x}}\biggr)\phi(x)
\end{align*}

Where $\frac{\partial}{\partial\vec{x}}$ denotes the gradient operator.
