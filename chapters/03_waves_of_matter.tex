\chapter{Waves of matter}

\section{The Shr\"odinger equation}
Physical laws can never be demonstrated, but only inferred from experiment and then verified or falsified by other ones


  \subsection{Double-slit experiment}
  The propagation of the electrons in the double-slit experiment has wave-like properties.
  The probability of an electron being detected in certain points of the screen is described by complex-wave apliduteds, which are functions of $x$ and $t$, so being $A(x,y)$ a wave, its intensity:

  $$Intensity(x,t) = A^*(x,t)\cdot A(x,t)$$

  The estate of the electron in the beams is assumed described by a complex amplitude called the wave-function:

  \begin{align*}
    \phi(x,t) &\rightarrow Prob(x,t)\\
              &=\phi^*(x,t)\phi(x,t)\\
              &\equiv |\phi(x,t)|^2
  \end{align*}

  \subsection{Main assumption}
  Electrons behave exactly like photons: they both have a dual particle and wave nature.
  So for electron too is assumed that energy is proportional to frequency:

  $$E = \hbar\omega = h\nu$$

  Protons propagate according to a wave equation:

  $$\biggl(\underbrace{\frac{1}{c^2}}_{\text{speed of light}}\frac{\partial^2{}}{\partial{t^2}} - \nabla^2\biggr)\underbrace{\text{Maxwell equation}} = 0$$

  There is a need to find if the same equation describes the propagation of a massive particle like the electron.
  Let the electron wave particle:

  $$\phi(x,t) = Ae^{i(\vec{k}\vec{r}-\omega t)}$$

  Then: $\frac{\partial^2{}}{\partial{t^2}}\phi = \omega^2\phi$ and $\nabla^2\phi = |\vec{k}|^2\phi$.
  So it is obtained that: 

  $$\underbrace{k^2}_{\propto p^2}\propto\underbrace{\omega^2}_{\propto E^2}$$

  Then $E = cons|p|$.
  This is true for light, but it is false for massive particles: the correspondence principle implies that:

  $$E = \frac{p^2}{2m}\rightarrow E\propto p^2$$

  And not just to $|p|$.

  \subsection{Defining the Shr\"oedinger equation}
  So there is a need to use a different wave equation with respect to the photon's one.
  Noticing that $E^2\propto p^2$ follows from having two time derivatives, to have $E\propto p^2$ one time derivative is tried:

  $$\frac{\partial {}}{\partial {t}}e^{i(\vec{k}\vec{r}-\omega t)} = -i\omega e^{i(\vec{k}\vec{r}-\omega t)}$$

  And:

  $$\biggl(iA \frac{\partial {}}{\partial {t}}+ B\nabla^2\biggr)\phi = 0\Rightarrow \omega A - B k^2 = 0$$

  To obtain an energy $A = \hbar$, then:

  $$\hbar\omega = E = Bk^2\propto p$$

  The remaining constant is set by phenomenology:

  $$i\hbar \frac{\partial {}}{\partial {t}}\phi = -\frac{\hbar^2}{2m}\nabla^2\phi$$

  The right hand side represent a kinetic energy.
  For a free electron $H$ equals the kinetic energy and is approximately the Hamiltonian.
  For an interacting electron:

  $$H_0 \rightarrow H_0 + \underbrace{V(\vec{r})}_{\text{potential energy}} = H$$

  So finally the Schr\"oedinger equation:

  $$i\hbar \frac{\partial {}}{\partial {t}}\phi(\vec{r},t) = \biggl(-\frac{\hbar^2}{2m}\nabla^2+V(\vec{r})\biggr)\phi(x,t)$$

    \subsubsection{Quantum Hamiltonian}
    Now the quantum Hamiltonian can be defines as:

    $$\hat{H} \equiv -\frac{\hbar^2}{2m}\nabla^2+U(\vec{r})$$

    The Schr\"oedinger equation is then recast as:

    $$i\hbar \frac{\partial {}}{\partial {t}}\phi(\vec{r},t) = \hat{H} \phi(\vec{r},t)$$

\section{Stationary Shr\"oedinger equation}
Assuming that the system conserves mechanical energy:

$$H(p,q, \bcancel{t}) = \frac{p^2}{2m}+U(q,\bcancel{t})$$

Considering the plane wave:

$$\phi(\vec{r}, t) = \psi(\vec{r}) e^{-i \frac{E}{\hbar}t}$$

As a guess for the form of the solution and plugging it into the Schr\"oedinger equation:

$$E\psi e^{-i \frac{i}{\hbar}(Et)} = He^{-\frac{i}{\hbar}Et}\psi$$

Which gives:

$$\hat{H}\phi(\vec{r}) = E\psi(\vec{r})$$

Where $E$ is unknown.
This is an eigen problem: given $\hat{H}$ a function $\psi(\vec{r}($ has to be found such that $\hat{H}\psi(\vec{r})$ is a function proportional to $\psi(\vec{r})$ through some constant $E$.
In other words the energy of a quantum system is an eigenvalue of the quantum Hamiltonian operator.
So, from a discrete spectrum energy quantization is obtained and:

$$E = \underbrace{E_0}_{\text{ground state energy}} < \underbrace{E_1 < \cdots < E_n}_{\text{excited state energies}}$$

And:

\begin{align*}
  \hat{H}\psi(\vec{r}) &= E_0\psi_0(\vec{r})\\
  \hat{H}\psi(\vec{r}) &= E_1\psi_1(\vec{r})\\
                       &\vdots\\
  \hat{H}\psi(\vec{r}) &= E_n\psi_n(\vec{r})\\
\end{align*}

\section{Ground state of the quantum harmonic oscillator}
The quantum Hamiltonian for the quantum harmonic oscillator is:

$$\hat{H} = - \frac{\hbar^2}{2m}\frac{d{^2}}{d{x^2}} +\frac{1}{2}m\omega^2 x^2$$

The solution for the ground state should be in the form $\hat{H}\psi_0 = E$.

  \subsection{Form of the ground state}
  The ground state will be in the form:

  $$\phi(x) = \mathcal{N}e^{-\frac{\alpha x^2}{4}}$$

  So:

  \begin{align*}
    -\frac{\hbar^2}{2m}\frac{d{^2}}{d{x^2}}\psi_0 &= -\frac{\hbar^2}{2m}\mathcal{N}\frac{d{}}{d{x}}\biggl[-\frac{\alpha}{2}xe^{-\frac{\alpha x^2}{4}}=\\
                                                  &= -\frac{\hbar^2}{2m}\biggl[-\frac{\alpha}{2}e^{-\frac{\alpha x^2}{4}}+\frac{\alpha^2}{4}x^2e^{-\frac{\alpha x^2}{4}}\biggr]\mathcal{N}
  \end{align*}

  So:

  \begin{align*}
    \hat{H}\psi_0 &= -\frac{\hbar^2}{2m}\frac{d{^2}}{d{x^2}}\psi_0 + \frac{1}{2}m\omega^2x^2\psi_0=\\
                  &= \frac{\alpha}{4}\frac{\hbar^2}{m}\psi_0 -\frac{\hbar^2\alpha^2}{8m}x^2\psi_0 + \frac{1}{2}m\omega^2x^2\psi_0
  \end{align*}

  For $\psi_0$ to be an eigenstate $\hat{H}\psi_0 = const\cdot\psi_0$ so the dependence on $x^2$ must be cancelled:

  $$-\frac{\hbar^2}{\bcancel{2}m}\alpha^2\cancelto{2}{4}\bcancel{x^2}\bcancel{\psi_0} + \frac{1}{2}m\omega^2\bcancel{x^2}\bcancel{\psi_0} = 0$$

  From this:

  $$\alpha^2 = \frac{m^2\omega^2}{4\hbar^2}\Rightarrow\alpha=\frac{m\omega}{2\hbar}$$

  So:

  $$\psi_0(x) = \mathcal{N}e^{-\frac{m\omega}{2\hbar}x^2}$$

\section{Quantum particle in a one dimensional infinite square well}
For the case of a quantum particle in a one dimensional infinite square well consider a particle constrained in a trap where interactions are so strong that it cannot escape and with two confining direction much narrower than the third.
This can be modelled with a one dimensional confining potential:

$$U(x) = \begin{cases} 0, &-\frac{L}{2}<x<\frac{L}{2}\\\infty &|x|>\frac{L}{2}\end{cases}$$

Inside the box $U=0$, so the Schr\"oedinger equation is:

$$-\frac{\hbar^2}{2m}\frac{d{^2}}{d{x^2}}\psi(x) = E\psi(x)$$

Mathematically the looks like the Newton's equation for the harmonic oscillator:

$$+m \frac{d{^2}}{d{t^2}}x(t) = -kx(t)$$

Where $x\rightarrow t$ and $\psi\rightarrow x$.
However the constraints are different, instead of the classical initial values $x(0) = x_0$ and $\frac{d{}}{d{t}}x|_{t=0} = 0$ there is a boundary value $\psi\biggl(\pm \frac{L}{2}\biggr) = 0$.

  \subsection{Solution}
  Given the mathematical similarity between the two equation the general structure of the solution should be:

  $$\psi(x) = \begin{cases}A_1\cos k_1 x \\A_2\sin k_2x\end{cases}$$

  Where $A_1,A_2,k_1, k_2$ need to be fixed.

    \subsubsection{Option 1}
    $A_1\cos\biggl(k_1 \frac{L}{2}\biggr) = 0$, then:

    \begin{align*}
      k_1 \frac{L}{2} &=\pm \frac{\pi}{2}\pm n\pi\Rightarrow\\
      \Rightarrow k_1^{(n)} &=\pm \frac{\pi}{L},\pm \frac{3\pi}{L},\dots =\\
                            &= \pm 2(n-1)\frac{\pi}{L}\qquad n = \mathbb{N}
    \end{align*}

    So for this option the possible quantized energy levels are:

    $$E^{(n)} = \frac{\hbar^2}{2m}\frac{\pi^2}{L^2}(2n-1)^2$$

    \subsubsection{Option 2}
    $A_1\sin\biggl(k_2 \frac{L}{2}\biggr) = 0$, then:

    \begin{align*}
      k_2 \frac{L}{2} &=\pm \pi\pm n\pi\Rightarrow\\
      \Rightarrow k_2^{(n)} &=\pm \frac{2\pi}{L},\pm \frac{4\pi}{L},\dots =\\
                            &= \pm 2n\frac{\pi}{L}\qquad n = \mathbb{N}
    \end{align*}

    So for this option the possible quantized energy levels are:

    $$E^{(n)} = \frac{\hbar^2}{2m}\frac{\pi^2}{L^2}(2n)^2$$

    \subsubsection{$A_1$ and $A_2$}
    $A_1$ and $A_2$ can be determined by the normalization conditions.
    The probability of finding the particle somewhere in the box is one, so:

    \begin{align*}
      \int_{-\frac{L}{2}}^{\frac{L}{2}} dx|\psi(x)|^2 = 1\\
      1 = \int_{-\frac{L}{2}}^{\frac{L}{2}} dx\begin{cases}A_1^2\cos^2k_1x\\A_2^2\sin^2 k_2x\end{cases}\Rightarrow\begin{cases}A_1 = \\A_2 = \end{cases}
    \end{align*}

    \subsubsection{Summary}
    To summarize the results:

      \paragraph{Energy spectrum}

      $$E_n = \frac{\hbar^2}{2m}\frac{\pi^2}{L^2}\mathcal{L}^2$$

      \paragraph{Energy eigenstates}
      The energy eigenstates of the stationary wave-functions are:

      $$\psi_m(x) = \begin{cases}A_1\cos k_1 x & n\ odd\\ A_2\sin k_2 x & n\ een\end{cases}$$


  \subsection{Discussion}

    \subsubsection{Quantized momenta}
    From the quantized energy:

    $$E_n = \frac{1}{2m}\biggl(\underbrace{\frac{\pi^2\hbar^2}{L_2}m^2}_{=p_m^2}\biggr)$$

    So the quantized momenta is $p_m = \pm \frac{\pi\hbar}{L}m$.

    \subsubsection{Lowest energy state}
    In classical mechanics the lowest energy state is $p = 0\Rightarrow E = 0$.
    However in quantum mechanics the lowest energy level is:

    $$E_1 = \frac{\hbar^2}{2m}\biggl(\frac{\pi}{L}\biggr)^2 > 0$$

    Recalling that $E_1 = \frac{p_1^2}{2m}$, it is inferred that:

    $$p_1 = \pm \frac{\hbar\pi}{L}\neq 0$$

    $p_1$ can point in $+$ or $-$ direction with equal probability, so the quantum uncertainty is $\Delta p =\frac{2\hbar\pi}{L}$.
    On the other hand $\Delta p \sim L$ so there is quantum delocalization for $\psi_1(x)$.
    However in the classical ground state:

    $$T = \frac{p^2}{2m = 0}\Rightarrow p = 0\Rightarrow \Delta p = 0$$

    But $\Delta q = b$, hence there is a violation of the uncertainty principle:

    $$\Delta q\Delta p = 0$$.

    \subsubsection{Correspondence principle}
    Considering for $L\rightarrow\infty$ and $m\rightarrow\infty$ $E_0 = 0$, so there is no uncertainty in classical mechanics and $E_{n+1}-E_n \rightarrow 0$, so there is no quantization.
    So classical mechanics is contained in quantum mechanics in the macroscopic limit, for large size and heavy masses.

    \subsubsection{Excited states}
    The wave function of the n-th excited state has $N$ nodes, a general result that holds for any quantum system.

\section{Two dimensional square well}
Consider a quantum particle in a two dimensional square well with dimension $L_1$ and $L_2$.
Then:

$$\hat{H} = -\frac{\hbar^2}{2m}\frac{\partial {^2}}{\partial {x^2}}-\frac{\hbar^2}{2m}\frac{\partial {^2}}{\partial {y^2}}$$

Any $H$ can be separated as: $H(x,y) = H_1(x) + H_2(y)$.

  \subsection{Solution}
  Then the solution is:

  $$\begin{cases}\psi(x,y) = \psi_1(x)\psi_2(y)\\H\psi=(E_1+E_2(\psi))\end{cases}$$

  Where $H_1\psi_1 = E_1\psi_1$  and $H_2\psi_2 = E_2\psi_2$
  So:

  \begin{align*}
    (H_1 + H_2)\psi_1(x)\psi_2(y) &= H_1\psi_1(x)\psi_2(y) + H_2\psi_1(x)\psi_2(y)=\\
                                  &= \psi_2(y)\hat{H}_1\psi_1(x)+\psi_1(x0+\psi)1(x)H_2\psi_2(y)=\\
                                  &= \psi_2 E_1\psi_1 + \psi_1E_2\psi_2=\\
                                  &=(E_1+E_2)\psi_1\psi_2
  \end{align*}

  So:

  $$\phi_{n,m}(x,y) = \underbrace{\psi_n(x)}_{\text{solution of the 1D problem}}\psi_m(y)$$

  And $E_{n,m} = E_n+E_m$.

\section{Lattice discretization}
Lattice discretization is a technique by which a numerical solution is obtained exploiting the connection between operators and matrices.
A large but finite number of equally spaced possible position is used.
The possible positions are $x = x_{\min} + \Delta x$, where $\Delta x = \frac{x_{\max}-x_{\min}}{N}$.
Then the wave function is represented by a list $\psi(x) \rightarrow(\psi_1,\dots, \psi_N)$.
Thus the wave function becomes  a vector and the Hamiltonian a matrix.

  \subsection{Discrete representation of derivative}

  $$\frac{d{}}{d{x}}\psi(x) \rightarrow \frac{\psi_{i+1}-\psi_i}{\Delta x}$$

  $$\frac{d{^2}}{d{x^2}}\psi(x) \rightarrow \frac{\psi_{i+1}+\psi_{i-1}-2\psi_i}{\Delta x^2}$$

  Let the Kronecker delta be:

      $$\delta_{ij} =\begin{cases}1 &i=j\\0 &i\neq j\end{cases}\Rightarrow S_{ij} \rightarrow\begin{pmatrix}1 & 0 & \cdots\\\vdots & \ddots &\vdots \\0&\cdots & 1 \end{pmatrix}$$

  Now:

  $$-\frac{\hbar^2}{2m}\frac{d{^2}}{d{x^2}}\rightarrow T_{ij}=-\frac{\hbar^2}{2m\Delta x^2}(\delta_{ij+1}+\delta_{ij-1}-2\delta_{ij})$$

  Indeed:

  $$\sum\limits_{j}T_{ij}\psi_i = -\frac{\hbar^2}{2m}(\psi_{i+1}+\psi_{i-1}-2\psi_i)\frac{1}{\Delta x^2}$$

  Now $V(x) \rightarrow V_{ij} = V_i\delta_{ij}$, so:

  $$\sum\limits_{j}V_{ij}\psi_j = V_i\psi_i$$

  Finally:

  $$H_{ij} = T_{ij} + V_{ij}$$

  $$H_{ij} = \begin{pmatrix}\frac{\hbar^2}{m\Delta x^2} + V_1 & -\frac{\hbar^2}{2m\Delta x} & 0 & \cdots\\ -\frac{\hbar^2}{2m\Delta x^2} & \frac{\hbar^2}{m\Delta x^2}+V_2-\frac{\hbar^2}{2m \Delta x^2} & 0 & \cdots\\ 0 & -\frac{\hbar^2}{2m\Delta x^2} & \frac{\hbar^2}{m\Delta x^2}+V_3 - \frac{\hbar^2}{2m\Delta x^2} & \cdots\end{pmatrix}$$

  Solving the  matrix eigenproblem $\sum\limits_i H_{ij}\psi_i = E_i\psi_i$ yields $\{E_i\}_{i=\{1,\dots,N\}}$ and $\{y_j\}_i$.
  The dimensionality grows with the number of mesh points $N$.
  The exact case is $N\rightarrow\infty$.
  Quantum mechanics is described by a infinite dimensional vector space called the Hilbert space.
