\chapter{Waves of matter}

\section{The Shr\"odinger equation}
Physical laws can never be demonstrated, but only inferred from experiment and then verified or falsified by other ones


  \subsection{Double-slit experiment}
  The propagation of the electrons in the double-slit experiment has wave-like properties.
  The probability of an electron being detected in certain points of the screen is described by complex-wave apliduteds, which are functions of $x$ and $t$, so being $A(x,y)$ a wave, its intensity:

  $$Intensity(x,t) = A^*(x,t)\cdot A(x,t)$$

  The estate of the electron in the beams is assumed described by a complex amplitude called the wave-function:

  \begin{align*}
    \phi(x,t) &\rightarrow Prob(x,t)\\
              &=\phi^*(x,t)\phi(x,t)\\
              &\equiv |\phi(x,t)|^2
  \end{align*}

  \subsection{Main assumption}
  Electrons behave exactly like photons: they both have a dual particle and wave nature.
  So for electron too is assumed that energy is proportional to frequency:

  $$E = \hbar\omega = h\nu$$

  Protons propagate according to a wave equation:

  $$\biggl(\underbrace{\frac{1}{c^2}}_{\text{speed of light}}\frac{\partial^2{}}{\partial{t^2}} - \nabla^2\biggr)\underbrace{\text{Maxwell equation}} = 0$$

  There is a need to find if the same equation describes the propagation of a massive particle like the electron.
  Let the electron wave particle:

  $$\phi(x,t) = Ae^{i(\vec{k}\vec{r}-\omega t)}$$

  Then: $\frac{\partial^2{}}{\partial{t^2}}\phi = \omega^2\phi$ and $\nabla^2\phi = |\vec{k}|^2\phi$.
  So it is obtained that: 

  $$\underbrace{k^2}_{\propto p^2}\propto\underbrace{\omega^2}_{\propto E^2}$$

  Then $E = cons|p|$.
  This is true for light, but it is false for massive particles: the correspondence principle implies that:

  $$E = \frac{p^2}{2m}\rightarrow E\propto p^2$$

  And not just to $|p|$.

  \subsection{Defining the Shr\"oedinger equation}
  So there is a need to use a different wave equation with respect to the photon's one.
  Noticing that $E^2\propto p^2$ follows from having two time derivatives, to have $E\propto p^2$ one time derivative is tried:

  $$\frac{\partial {}}{\partial {t}}e^{i(\vec{k}\vec{r}-\omega t)} = -i\omega e^{i(\vec{k}\vec{r}-\omega t)}$$

  And:

  $$\biggl(iA \frac{\partial {}}{\partial {t}}+ B\nabla^2\biggr)\phi = 0\Rightarrow \omega A - B k^2 = 0$$

  To obtain an energy $A = \hbar$, then:

  $$\hbar\omega = E = Bk^2\propto p$$

  The remaining constant is set by phenomenology:

  $$i\hbar \frac{\partial {}}{\partial {t}}\phi = -\frac{\hbar^2}{2m}\nabla^2\phi$$

  The right hand side represent a kinetic energy.
  For a free electron $H$ equals the kinetic energy and is approximately the Hamiltonian.
  For an interacting electron:

  $$H_0 \rightarrow H_0 + \underbrace{V(\vec{r})}_{\text{potential energy}} = H$$

  So finally the Schr\"oedinger equation:

  $$i\hbar \frac{\partial {}}{\partial {t}}\phi(\vec{r},t) = \biggl(-\frac{\hbar^2}{2m}\nabla^2+V(\vec{r})\biggr)\phi(x,t)$$

    \subsubsection{Quantum Hamiltonian}
    Now the quantum Hamiltonian can be defines as:

    $$\hat{H} \equiv -\frac{\hbar^2}{2m}\nabla^2+U(\vec{r})$$

    The Schr\"oedinger equation is then recast as:

    $$i\hbar \frac{\partial {}}{\partial {t}}\phi(\vec{r},t) = \hat{H} \phi(\vec{r},t)$$

\section{Stationary Shr\"oedinger equation}
Assuming that the system conserves mechanical energy:

$$H(p,q, \bcancel{t}) = \frac{p^2}{2m}+U(q,\bcancel{t})$$

Considering the plane wave:

$$\phi(\vec{r}, t) = \psi(\vec{r}) e^{-i \frac{E}{\hbar}t}$$

As a guess for the form of the solution and plugging it into the Schr\"oedinger equation:

$$E\psi e^{-i \frac{i}{\hbar}(Et)} = He^{-\frac{i}{\hbar}Et}\psi$$

Which gives:

$$\hat{H}\phi(\vec{r}) = E\psi(\vec{r})$$

Where $E$ is unknown.
This is an eigen problem: given $\hat{H}$ a function $\psi(\vec{r}($ has to be found such that $\hat{H}\psi(\vec{r})$ is a function proportional to $\psi(\vec{r})$ through some constant $E$.
In other words the energy of a quantum system is an eigenvalue of the quantum Hamiltonian operator.
So, from a discrete spectrum energy quantization is obtained and:

$$E = \underbrace{E_0}_{\text{ground state energy}} < \underbrace{E_1 < \cdots < E_n}_{\text{excited state energies}}$$

And:

\begin{align*}
  \hat{H}\psi(\vec{r}) &= E_0\psi_0(\vec{r})\\
  \hat{H}\psi(\vec{r}) &= E_1\psi_1(\vec{r})\\
                       &\vdots\\
  \hat{H}\psi(\vec{r}) &= E_n\psi_n(\vec{r})\\
\end{align*}

\section{Ground state of the quantum harmonic oscillator}

\section{Quantum particle in a one dimensional infinite square well}

  \subsection{Energy spectrum}

  \subsection{Energy eigenstates}

  \subsection{Discussion}

\section{Two dimensional square well}

\section{Lattice discretization}


