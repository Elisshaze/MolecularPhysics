\chapter{Partial derivatives}

\section{First order derivatives}
The concept of derivative can be used to explore functions of $n\ge 2$ variables.
Let $f:\mathbb{R}^2\supseteq A\rightarrow \mathbb{R}$, where $A$ is an open set of $\mathbb{R}^2$, be a function of two variables: $f(x, y)$.
The partial derivative of $f(x,y)$ with respect to $x$ in the point $(x_0, y_0)$ is defined as:

$$\frac{\partial f(x_0, y_0)}{\partial x} := \lim\limits_{h\rightarrow 0}\frac{f(x_0+h, y_0) -f(x_0, y_0)}{h}$$

With $h\in\mathbb{R}$ when the limit exists.
Equivalently the partial derivative of $f(x, y)$ with respect to $y$ in $(x_0, y_0)$ is:

$$\frac{\partial f(x_0, y_0)}{\partial y} := \lim\limits_{h\rightarrow 0}\frac{f(x_0, y_0+h) -f(x_0, y_0)}{h}$$

With $h\in\mathbb{R}$ when the limit exists.
That is, the derivative of $f(x,y)$ with respect to a variable is computed as if other variables are held constant.
The existence of the partial derivative with respect to one variable does not imply the existence of the partial derivatives along any other direction.
The derivative along a general direction $\vec{v}$ is called directional derivative and is defined as:

$$D_{\vec{v}}f(x_0, y_0) := \lim\limits_{t\rightarrow 0}\frac{f((x_0, y_0) + t\vec{v}) - f(x_0, y_0)}{t}$$

With $t\in\mathbb{R}$ where the limit exists.

	\subsection{Differentiability}
	The concept of differentiability is introduced because the existence of the derivative along one direction does not imply the existence of directional derivatives along different directions.
	Let $\mathbb{R}^2\supseteq A\rightarrow\mathbb{R}$, with $A$ an open set of $\mathbb{R}^2$; a function of two variables $f(x,y)$ is differentiable if the partial derivatives exist in $(x_0, y_0)$ and:

	$$\lim\limits_{(h,k)\rightarrow(0,0)}\frac{f(x_0+h, y_0+k) - f_x(x_0, y_0)h - f_y(x_0, y_0)k}{\sqrt{h^2+k^2}} = 0$$

	Where $f_x$ and $f_y$ are the partial derivatives with respect to $x$ or $y$.

	\subsection{Tangent plane}
	The tangent plane of $f(x,y)$ in the point $(x_0, y_0)$ has the following form:

	$$g(x,y) = f(x_0, y_0) + f_x(x_0, y_0)(x-x_0) + f_y(x_0, y_0)(y-y_0)$$

	\subsection{Determine if a function is differentiable}
	A function if differentiable in a point if the following condition holds true.
	Let $f:\mathbb{R}^2\supseteq A\rightarrow \mathbb{R}$ with $A$ an open set of $\mathbb{R}^2$.
	If in a neighbourhood of $(x_o, y_0)$ all the partial derivatives of $f(x, y)$ exist and are continuous in $(x_0, y_0)$ then $f(x,y)$ are differentiable in $(x_0, y_0)$.
	If a function has all the partial derivatives in a point and they are continuous, the function is differentiable.
	That means that the tangent plane in that point exists.

\section{Higher order derivatives}
Let $f:\mathbb{R}^2\supseteq A\rightarrow\mathbb{R}$, with $A$ an open set of $\mathbb{R}^2$, a function of two variables $f(x,y)$.
Supposing that the partial derivatives exist in a neighbourhood $I$ of $(x_0, y_0)$, the two functions $g(x,y) = \frac{\partial f(x,y)}{dx}:\mathbb{R}^2\supseteq I\rightarrow\mathbb{R}$ and $h(x,y) = \frac{\partial f(x,y)}{dy}:\mathbb{R}^2\supseteq I\rightarrow\mathbb{R}$ can be seen as the analogous of $f$ and there is a possibility of taking the partial derivatives of $g$ and $h$ in a point $(x_0, y_0)$.
This means applying the $g$ and $h$ the operators $\frac{\partial }{\partial x}$ and $\frac{\partial}{\partial y}$.
The second order derivatives are defined as:

\begin{table}[H]
	\begin{tabular}{c c}
		$\frac{\partial^2 f}{\partial x^2}: \frac{\partial}{\partial x}g = \frac{\partial}{\partial x}\frac{\partial f}{\partial x}$ &
		 $\frac{\partial^2 f}{\partial y\partial x}: \frac{\partial}{\partial y}g = \frac{\partial}{\partial y}\frac{\partial f}{\partial x}$ \\
		$\frac{\partial^2 f}{\partial x\partial y}: \frac{\partial}{\partial x}h = \frac{\partial}{\partial x}\frac{\partial f}{\partial y}$ &
		 $\frac{\partial^2 f}{\partial y^2}: \frac{\partial}{\partial y}h = \frac{\partial}{\partial y}\frac{\partial f}{\partial y}$\\
	\end{tabular}
	\centering
\end{table}

When the partial derivative is taken two times in the same direction the second partial derivatives are named pures, when they are taken along a different direction with respect to the first time they are named mixed.

	\subsection{Schwartz's theorem}
	Let $f(x, y)$ be a function defined in $\mathbb{R}^2$ and $I$ a neighbourhood of $(x_0, y_0)$ and $\partial x\partial y f$ and $\partial y\partial x f$ be continuous in $I$, then:

	$$\frac{\partial^2 f}{\partial x\partial y} = \frac{\partial ^2 f}{\partial y\partial x}$$

	All of this can be extended to higher order partial derivatives and to functions from $\mathbb{R^n}$ to $\mathbb{R}$ with an increasing number of combinations of derivatives.
	This theorem is useful for many reasons, one of which is the fact that if the fourth order mixed partial derivatives are continuous in $(x_0, y_0)$ the order of the first order partial derivatives can be rearranged as preferred.

\section{Differential equation}
A differential equation is a relation between an unknown function $f(\vec{x})$ and its arbitrary-order derivatives valid for every point $\vec{x}$ of the domain under consideration.
The general solutions of differential equations involves several arbitrary constants, depending on the type of the equation and on the order of the derivatives involved.
The general solution of a partial differential equation involves an infinite set of unknown constants.
Obtaining a particolar solution involves the addition of boundary or initial condition.
There are two types of differential equations: linear and non-linear.
The Schroedinger equation is a differential equation of the first order in time and second order in coordinates and a linear partial differential equation.
