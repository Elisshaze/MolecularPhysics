\chapter{Dirac notation}
\textit{Ref: D. J. Griffith - Introduction to Quantum Mechanics, second edition}\\

In quantum mechanics, the state of a system can be represented by a vector $\ket{\vec{L}(t)}$ that "lives out there in Hilbert space" but can be expressed by means of any number of different BASES.\\
The wave function $\Psi(x,t)$ is the coefficient in the expansion of $\ket{\vec{L}}$ in the basis of position eigenfunctions.
\[
\Psi(x,t)=\braket{x}{\vec{L}}
\]
The momentum space wavefunction $\Phi(p,t)$ is the expansion of $\ket{\vec{L}}$ in the basis of momentum eigenfunctions.
\[
\Phi(p,t)=\braket{p}{\vec{L}}
\]
We can also expand the wave function $\ket{\vec{L}}$ with energy eigenfunctions (discrete spectrum)
\[
c_n(t)=\braket{n}{\vec{L}}
\]
The functions $\Psi$, $\Phi$, and the set of coefficients ${c_n}$ are the same way of describing the same vector:
\[
\Psi(x,t)=\int\ Psi(y,t)\delta(x-y)\,dy\,=\int\Phi(p,t)\cdot\frac{1}{\sqrt{2\pi \hbar}}e^{\frac{ipx}{\hbar}}dp\,=\sum c_n\cdot e^{-\frac{iE_nt}{\hbar}}\psi_n(x)
\]
Operators acting on observables are linear transformations and transform one vector into another $\ket{\beta}=\hat{O}\ket{\alpha}$ and can be represented with respect to a particular basis by their matrix elements.
\[
\mel{e_m}{\hat{O}}{e_n}\equiv Q_{mn} \text{ so we can say }
\sum_n b_n\ket{e_n}=\sum_na_n\hat{O}\ket{e_n}
\]
by considering the inner product $\sum_nb_n\braket{e_m}{e_n}=\sum_na_n\mel{e_m}{\hat{O}}{e_n}$
\[
b_m=\sum_nQ_{mn}a_n
\]
where $Q_{mn}$ is a matrix element that will determine how the components transform.\\
I might have a system that only admits a finite number of linearly independent states (usually two). $\ket{L(t)}$ lives in an N-dimensional vector space that can be represented with a column of N components. Operators take the form of a (NxM) matrix.\\
Anyway, for the inner product Dirac proposed the \textbf{braket notation} $\braket{a}{b}$ where:
\begin{itemize}
    \item $\ket{b}$ Ket = vector
    \item $\bra{a}$ Bra = linear function of vector that, if it hits the ket, it gives the inner product of the two, a \underline{number}
\end{itemize}
Operator + vector = vector\\
Bra + vector = number\\
Bra notation can be resembled as a function to integrate $\bra{f}=\int f^*[...]dx$ in an infinite vector space. In a finite vector space, the bra represents a row complex vector such as
\[
\braket{a}{b}=\begin{pmatrix}a^*_1&a^*_2&...&a^*_n\end{pmatrix}\begin{pmatrix}b_1\\b_2\\...\\b_n\end{pmatrix}
\]
Collection of bras is another vector space = DUAL SPACE\\
\section{Vector space representation}
In vector space ( $\mathbb{R}^3$), we can express each $v$ as $v_i=\hat{e}_i \cdot \vec{v}$. $\hat{e}_1,\hat{e}_2,\hat{e}_3$ are orthogonal and orthonormal. The choice of the basis is key.
\subsection{Inner product}
$$(\bar{v}\bar{w})=\sum_{i} v_i w_i$$
\subsection{Operator}
$$(\hat{e}_{j} ; \vec{0}(\sum_{i} \hat{e}_{i} v_{i})_{1}=0_{j i}$$
\begin{enumerate}
\item take each component of the vector
\item apply transformation
\item compute the product of the transformed components
\end{enumerate}

$$
\left(\hat{e}_{i j}(\hat{o} \bar{v})\right)=(\hat{e}_{j}, \sum_{i} \hat{o} \underbrace{\hat{e}_{i} v_{i}}_{v_{i}})= \\
=\sum({\left.\hat{e}_{j}, \hat{o} \hat{e}_{i}\right) v_{i}}$$
As a result: $$ \quad \hat{o} \bar{v} \rightarrow \Sigma_{j i} v_{i}=w_{j}$$

\section{Hilbert space representation}
The elements of the Hilbert space are represented as following: $\ket{\psi},\ket{\xi} \in \mathcal{H}$
\subsection{Inner product}
The Dirac delta function can assume either $0$ or $1$ as a value.
$$\bra{x}\ket{x} = \delta (x,x')$$
\noindent
We can compute $\ket{\psi}$ as the integral over the inner product with the position ($\bra{x}\ket{\psi} \in \mathbb{C}$) and the position vector itself.

$$\ket{\psi}= \int d x \bra{x}\ket{\psi} \ket{x} = \int d x \psi(x) \ket{x} $$
\subsection{Operator}
$$\mel{x}{\hat{p}}{\psi}=\int d x^{\prime} \underbrace{\left\langle x|p| x^{\prime}\right\rangle}_{\operatorname{matrix}} \psi\left(x^{\prime}\right)
=\int d x \delta\left(x, x^{\prime}\right)\left(-i \hbar \frac{d}{d x^{\prime}}\right) \psi\left(x^{\prime}\right)
=-i \hbar \frac{d}{d x} \psi(x)$$
Since
$$
\int d x^{\prime} \delta\left(x, x^{\prime}\right) g\left(x^{\prime}\right)=g(x)$$
In general, $\sum_{j} \delta_{i j} A_{j}=A_{i} $. In the case in which $i=2$:
 $$ \underbrace{\delta_{21} A_{1}}_{0}+\underbrace{\delta_{22} A_{2}}_{1}+\underbrace{\delta_{23} A_{3}}_{0}
$$
\noindent
$\ket{\psi(x)}$ is called "ket" and $\bra{\psi(x)}$ is called "bra"; together they form the norm of $\psi$,  also known as "bracket".\\
\\
\underline{Expectation value}:
$$\expval{\hat{O}}{\psi}= \int d x \expval{\hat{O}}{\psi} \psi(x)= \int d x^{\prime} \psi^{*}(x) \mel{x'}{\hat{O}}{x} \psi(x)$$

\underline{S.E.}:
$$ \ket{\psi(t)}\rightarrow i \hbar \frac{d}{d t}\ket{\psi(t)}=\hat{H}\ket{\psi(t)}\stackrel{x \text { space }}{\rightarrow} i \hbar \frac{d}{d t}\bra{x} \ket{\psi(t)}=\int d x^{\prime} \mel{x'}{\hat{H}}{x} \bra{x} \ket{\psi}$$

\section{\textit{Example: two linearly independent states}}
\[\ket{1}=\begin{pmatrix}1\\0\end{pmatrix}\,\ket{2}=\begin{pmatrix}0\\1\end{pmatrix}\]
The most general state is a normalized linear combination: $\ket{L}=a\ket{1}+b\ket{2}=\begin{pmatrix}a\\b\end{pmatrix}$ with $|a|^2+|b|^2=1$.\\
\[\text{Hamiltonian: } \hat{H}=\begin{pmatrix}h&g\\g&h\end{pmatrix}, h,g\,\in \mathbb{R}\]
If the system at $t=0$ is at $\ket{1}$, what state will it be at $t$?
\begin{enumerate}
    \item Solve time independent Schr\"odinger equation $\hat{H}\ket{L}=E\ket{L}$ and find eigenvalues and eigenvectors.
    \[
    \det\begin{pmatrix}h-E&g\\g&h-E\end{pmatrix}=(h-E)^2-g^2=0 \rightarrow h-E=\pm g \rightarrow E_{\pm}=h\pm g
    \]
    I have two allowed energies.
    \item Determine eigenvectors:
    \[
    \begin{pmatrix}h&g\\g&h\end{pmatrix}\begin{pmatrix}\alpha\\\beta\end{pmatrix}=(h\pm g)\begin{pmatrix}\alpha\\\beta\end{pmatrix} \rightarrow h\alpha + g\beta = (h\pm g)\alpha \rightarrow \beta = \pm\alpha
    \]
    Normalized eigenvectors: $\ket{L_{\pm}}=\frac{1}{\sqrt{2}}\begin{pmatrix}1\\\pm 1\end{pmatrix}$
    \item Expansion of the initial state as linear combination of eigenvectors of the hamiltonian (entanglement of the two states).
    \[
    \ket{L(0)}=\begin{pmatrix}1\\0\end{pmatrix}=\frac{1}{\sqrt{2}}(\ket{L_+}+\ket{L_-})
    \]
    \item We add the time-dependence $e^{-\frac{iE_nt}{\hbar}}$ to calculate the state at time $t$:
    \[
    \ket{L(t)}=\frac{1}{\sqrt{2}}\bigg[e^{-\frac{i(g+h)t}{\hbar}}\ket{L_+}+e^{-\frac{i(h-g)t}{\hbar}}\ket{L_-}\bigg]
    \]
    \[
    =\frac{1}{2}\bigg[e^{-\frac{i(g+h)t}{\hbar}}\begin{pmatrix}1\\1\end{pmatrix}+e^{-\frac{i(h-g)t}{\hbar}}\begin{pmatrix}1\\-1\end{pmatrix}\bigg]
    \]
    \[
    =\frac{1}{2}e^{-\frac{iht}{\hbar}}\bigg[\begin{pmatrix}e^{-\frac{igt}{\hbar}}+e^{-\frac{igt}{\hbar}}\\e^{-\frac{igt}{\hbar}}-e^{-\frac{igt}{\hbar}}\end{pmatrix}\bigg]
    =e^{-\frac{iht}{\hbar}}
    \begin{pmatrix}
    \cos (gt/\hbar)\\
    -i\sin (gt/\hbar)
    \end{pmatrix}
    \]
\end{enumerate}
We can check this by observing that the result satisfies the time-dependent Schr\"odinger equation. It's a first raw model for neutrino oscillations :)
